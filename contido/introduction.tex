% SPDX-FileCopyrightText: 2025 Jorge Teixeira Crespo <jorge.teixeira@udc.es>
%
% SPDX-License-Identifier: GPL-2.0-or-later

\chapter{Introduction}
\label{chap:introduction}

\lettrine{I}{n} the current digital era, nearly every organization depends on a solid technological infrastructure to operate effectively. From internal coordination to external communication, from document management to compliance with legal obligations, many day-to-day tasks rely on tools hosted online or managed digitally. This trend has made IT infrastructure not only a convenience but a fundamental requirement.

For non-profit associations and student communities, the challenge is particularly complex. These entities often lack dedicated IT staff and operate with minimal budgets, yet they are expected to manage technological services. Moreover, many of them, especially those driven by ideological commitments, reject proprietary or SaaS solutions in favor of Free and Open Source Software (FOSS). This decision aligns with their values but often increases the burden of system maintenance and security.

A typical association today might need to run a website, handle mailing lists, share files securely, issue invoices under national tax regulations, and coordinate internal communication. These needs must be met through tools that are easy to maintain, well-documented, and adaptable to the association's size and structure. However, in practice, infrastructure is often built incrementally over many years by different people, leading to heterogeneous, outdated, and fragile systems.

GPUL (Grupo de Programadores e Usuarios de Linux) \cite{gpul_web} is one such organization, but with characteristics that make its case especially notable. Founded in 1998 in A Coruña, GPUL promotes Free Software and technical education through community events, workshops, and conferences. Closely tied to the field of computer engineering, many of its members are students or professionals with a strong technical background. This makes it easier to adopt and maintain open-source infrastructure, and also turns the infrastructure itself into a valuable learning environment. The association's long-standing commitment to the principles of software freedom drives it to self-host most of its services using libre tools, an approach that aligns with its values but also amplifies the common challenges of volunteer-managed systems.

Today, GPUL's services are distributed across two aging servers, GPULINO and GPULON, which run critical tools like an old Nextcloud instance for internal documentation and a Mailman 2 server that manages our mailing lists. These machines suffer from poor documentation, lack of backups, and limited monitoring, issues that are worsened by the fact that the board of the association changes every two years, often leaving gaps in technical knowledge. As current president, I've personally had to deal with several crises caused by this fragile setup.

One example was when our email redirection system, which uses manual redirects for addresses under the gpul.org domain, caused our messages to be flagged as spam, affecting communication with sponsors and participants during events. Another incident occurred shortly before the financial quarter deadline, when our Nextcloud instance went down due to a full disk caused by unrotated logs. Without documentation or clear monitoring, it took four days and help from former members and our hosting provider to bring the system back online, almost making us miss administrative deadlines with possible financial penalties.

Faced with this situation, we decided to renew GPUL's infrastructure entirely. The goal is not only to modernize the services but also to ensure they are well-documented, sustainable, and easy to maintain by future teams. In evaluating options, we considered managed services and commercial tools but discarded most of them due to their incompatibility with our philosophy or excessive costs. Among the various tools considered, Odoo \cite{odoo_web} stood out as a cloud-capable open-source platform that can also be self-hosted. We have chosen to self-host it and use it primarily for invoicing, particularly to comply with Spain's new \gls{verifactu} regulations \cite{BOE-A-2024-22138}. While invoicing is our main use case for now, Odoo's modular and extensible architecture leaves the door open to adopting other features in the future if needed.

For the rest of the stack, my approach has been to improve and modernize the existing FOSS tools already in use within the association. This includes migrating from Mailman 2 to Mailman 3 integrated with HyperKitty \cite{hyperkitty_web}, which improves usability and provides a web interface for list archives, and upgrading Nextcloud to a modern version with better backup and monitoring systems. For the email service, given its complexity and importance, we are still considering whether to keep it self-hosted or switch to a trusted external provider, always balancing autonomy with reliability.

This work documents the process of renewing GPUL's infrastructure with a focus on sustainability, maintainability, and alignment with our values. It includes an in-depth analysis of the current systems, the design of the new infrastructure, the rationale behind each technological choice, and the steps taken during migration. The aim is to leave behind a robust foundation and clear documentation for future contributors, ensuring that our commitment to Free Software is matched by reliable, modern, and resilient infrastructure.

\section{Objectives}

Given these critical challenges, the objectives of this project are to:

\begin{itemize}
    \item Modernize and simplify GPUL's infrastructure to ensure long-term maintainability.
    \item Replace outdated services (like Mailman 2 and legacy Nextcloud) with modern, modular open-source solutions.
    \item Implement proper documentation to facilitate knowledge transfer and reduce dependency on individual members.
    \item Enhance the reliability, performance, and security of critical infrastructure services, incorporating robust backup solutions.
    \item Empower new contributors with clear processes and documentation, ensuring continuity and sustainability.
    \item Ensure alignment with GPUL's foundational values of openness and community-driven development.
\end{itemize}

The priority lies in modernization, simplification, and thorough documentation, underpinned by improved security and reliable backup procedures.

\section{Thesis Structure}

TBD.
