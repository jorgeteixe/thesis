% SPDX-FileCopyrightText: 2025 Laura Milagros Castro Souto <lcastro@udc.es>
%
% SPDX-License-Identifier: GPL-2.0-or-later

%%%%%%%%%%%%%%%%%%%%%%%%%%%%%%%%%%%%%%%%%%%%%%%%%%%%%%%%%%%%%%%%%%%%%%%%%%%%%%%%

\pagestyle{empty}
\begin{abstract}
  This thesis studies the typical infrastructure needs of non-profit organizations
  and applies those insights to the renewal of GPUL's systems as a real-world case
  study. After analysing the limitations of two ageing servers, a modern
  architecture was designed around a dedicated Hetzner machine managed with Incus
  containers. Each required service was evaluated, the most suitable option was
  selected and then deployed. The resulting stack includes Mailcow for email,
  Mailman~3 for mailing lists, Nextcloud with Talk for groupware and meetings,
  Matrix for real-time communication, Passbolt for credential management and a
  PLG stack for monitoring and logging. All components are documented and
  integrated, providing a maintainable platform that ensures continuity for
  future volunteers.

  \vspace*{25pt}
  \begin{segundoresumo}
    Esta tese estuda as necesidades habituais de infraestrutura das asociacións
    sen ánimo de lucro e aplica esas conclusións ao caso práctico da renovación
    da infraestrutura de GPUL. Tras analizar o estado obsoleto de dous
    servidores, definiuse unha arquitectura moderna arredor dunha máquina
    dedicada en Hetzner xestionada con contedores Incus. Para cada servizo
    revisáronse as opcións existentes, seleccionáronse as máis axeitadas e
    despregáronse. O novo conxunto inclúe Mailcow para o correo, Mailman~3 para
    as listas, Nextcloud con Talk para o traballo colaborativo, Matrix para a
    comunicación en tempo real, Passbolt para a xestión de credenciais e un
    stack PLG para monitorización. Todos os compoñentes están documentados e
    integrados, proporcionando unha plataforma sostible que facilita a
    continuidade do labor voluntario.
  \end{segundoresumo}
\vspace*{25pt}
% SPDX-FileCopyrightText: 2025 Laura Milagros Castro Souto <lcastro@udc.es>
%
% SPDX-License-Identifier: GPL-2.0-or-later

\begin{multicols}{2}
\begin{description}
\item [\palabraschaveprincipal:] \mbox{} \\[-20pt]
  \blindlist{itemize}[7] % substitúe este comando por un itemize
                         % que relacione as palabras chave
                         % que mellor identifiquen o teu TFG
                         % no idioma principal da memoria (tipicamente: galego)
\end{description}
\begin{description}
\item [\palabraschavesecundaria:] \mbox{} \\[-20pt]
  \blindlist{itemize}[7] % substitúe este comando por un itemize
                         % que relacione as palabras chave
                         % que mellor identifiquen o teu TFG
                         % no idioma secundario da memoria (tipicamente: inglés)
\end{description}
\end{multicols}

\end{abstract}
\pagestyle{fancy}

%%%%%%%%%%%%%%%%%%%%%%%%%%%%%%%%%%%%%%%%%%%%%%%%%%%%%%%%%%%%%%%%%%%%%%%%%%%%%%%%
