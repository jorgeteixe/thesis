%%%%%%%%%%%%%%%%%%%%%%%%%%%%%%%%%%%%%%%%%%%%%%%%%%%%%%%%%%%%%%%%%%%%%%%%%%%%%%%%
% Preámbulo                                                                    %
%%%%%%%%%%%%%%%%%%%%%%%%%%%%%%%%%%%%%%%%%%%%%%%%%%%%%%%%%%%%%%%%%%%%%%%%%%%%%%%%

\documentclass[11pt,a4paper,titlepage,oneside]{report}

%%% RELACIÓN DE VARIABLES A PERSONALIZAR %%%
\def\lingua{gal}
%\def\lingua{esp} % descomenta esta liña se redactarás a memoria en español
%\def\lingua{eng} % descomenta esta liña se redactarás a memoria en inglés
\def\nome{Nome Completo}                             % substitúe aquí o teu nome
\def\nomedirectorA{Outro Nome Completo}              % substitúe aquí o nome de quen dirixe
%\def\nomedirectorB{Outro Nome Completo}             % duplica esta liña máis veces se o precisas, cambiando
                                                     % a letra final (A, B, C, D...): úsanse na portada.tex
\def\titulo{Título completo do traballo fin de grao} % substitúe aquí o título do teu TFG
%\def\titulacion{gced}                               % descomenta esta liña e comenta a seguinte se es estudante do GCED
\def\titulacion{gei}
%\def\mencion{COMPUTACIÓN}                           % descomenta a mención que che corresponda se es estudante do GEI
\def\mencion{ENXEÑARÍA DO SOFTWARE}
%\def\mencion{ENXEÑARÍA DE COMPUTADORES}
%\def\mencion{SISTEMAS DE INFORMACIÓN}
%\def\mencion{TECNOLOXÍAS DA INFORMACIÓN}

%\def\renomearcadros{si} % descomenta esta liña se redactas a memoria en español e prefires que
                         % os "cuadros" e o "índice de cuadros" se renomeen
                         % a "tablas" e "índice de tablas" respectivamente

\usepackage{estilo_tfg}

% Lista de paquetes potencialmente interesantes (uso baixo demanda)

% \usepackage{alltt}       % proporciona o entorno alltt, semellante a verbatim pero que respecta comandos
% \usepackage{enumitem}    % permite personalizar os entornos de lista
% \usepackage{eurofont}    % proporciona o comando \euro
% \usepackage{float}       % permite máis opcións para controlar obxectos flotantes (táboas, figuras)
% \usepackage{hhline}      % permite personalizar as liñas horizontais en arrays e táboas
  \usepackage{longtable}   % permite construir táboas que ocupan máis dunha páxina
% \usepackage{lscape}      % permite colocar partes do documento en orientación apaisada
% \usepackage{moreverb}    % permite personalizar o entorno verbatim
% \usepackage{multirow}    % permite crear celdas que ocupan varias filas da mesma táboa
% \usepackage{pdfpages}    % permite insertar ficheiros en PDF no documento
% \usepackage{rotating}    % permite diferentes tipos de rotacións para figuras e táboas
% \usepackage{subcaption}  % permite a inclusión de varias subfiguras nunha figura
% \usepackage{tabu}        % permite táboas flexibles
% \usepackage{tabularx}    % permite táboas con columnas de anchura determinada

%%%%%%%%%%%%%%%%%%%%%%%%%%%%%%%%%%%%%%%%%%%%%%%%%%%%%%%%%%%%%%%%%%%%%%%%%%%%%%%%
% Corpo                                                                        %
%%%%%%%%%%%%%%%%%%%%%%%%%%%%%%%%%%%%%%%%%%%%%%%%%%%%%%%%%%%%%%%%%%%%%%%%%%%%%%%%

\begin{document}

 %%%%%%%%%%%%%%%%%%%%%%%%%%%%%%%%%%%%%%%%
 % Preliminares do documento            %
 %%%%%%%%%%%%%%%%%%%%%%%%%%%%%%%%%%%%%%%%

 % SPDX-FileCopyrightText: 2025 Laura Milagros Castro Souto <lcastro@udc.es>
%
% SPDX-License-Identifier: GPL-2.0-or-later

\begin{titlepage}
  
  \hspace*{128pt}
  \textcolor{udcpink}{{\fontencoding{T1}\fontfamily{phv}\selectfont Facultade de Informática}}\\[-32pt]

  \begin{center}
    \includegraphics[scale=0.3]{imaxes/udc}\\[25pt]

    {\large TRABALLO FIN DE GRAO \\
            \nometitulacion \\
            \nomemencion } \\[10pt]

    \carimbo \\[25pt]

    \begin{huge}
      \begin{spacing}{1.3}
        \bfseries \titulo
      \end{spacing}
    \end{huge}
  \end{center}
  
  \vfill
  
  \begin{flushright}
    {\large
    \begin{tabular}{ll}
      {\bf Student:} & \nome \\
      {\bf Direction:} & \nomedirectorA \\
                     & \nomedirectorB \\ % duplica esta liña máis veces se o precisas, cambiando
                                           % a letra final (A, B, C, D...); define eses nomes no memoria_tfg.tex
    \end{tabular}}
  \end{flushright}
  \rightline{A Coruña, \datasimple.}
\end{titlepage}

 \dedicatoria{Dedicatoria} % escribe neste comando o teu texto de dedicatoria
 \paxinaenbranco
 \begin{agradecementos}
 \blindtext                % substitúe este comando polo teu texto de agradecementos
 \end{agradecementos}
 %%%%%%%%%%%%%%%%%%%%%%%%%%%%%%%%%%%%%%%%%%%%%%%%%%%%%%%%%%%%%%%%%%%%%%%%%%%%%%%%

\pagestyle{empty}
\begin{abstract}
  \blindtext % substitúe este comando polo resumo do teu TFG
             % na lingua principal do documento (tipicamente: galego)

  \vspace*{25pt}
  \begin{segundoresumo}
    \blindtext % substitúe este comando polo resumo do teu TFG
               % na lingua secundaria do documento (tipicamente: inglés)
  \end{segundoresumo}
\vspace*{25pt}
% SPDX-FileCopyrightText: 2025 Laura Milagros Castro Souto <lcastro@udc.es>
%
% SPDX-License-Identifier: GPL-2.0-or-later

\begin{multicols}{2}
\begin{description}
\item [\palabraschaveprincipal:] \mbox{} \\[-20pt]
  \blindlist{itemize}[7] % substitúe este comando por un itemize
                         % que relacione as palabras chave
                         % que mellor identifiquen o teu TFG
                         % no idioma principal da memoria (tipicamente: galego)
\end{description}
\begin{description}
\item [\palabraschavesecundaria:] \mbox{} \\[-20pt]
  \blindlist{itemize}[7] % substitúe este comando por un itemize
                         % que relacione as palabras chave
                         % que mellor identifiquen o teu TFG
                         % no idioma secundario da memoria (tipicamente: inglés)
\end{description}
\end{multicols}

\end{abstract}
\pagestyle{fancy}

%%%%%%%%%%%%%%%%%%%%%%%%%%%%%%%%%%%%%%%%%%%%%%%%%%%%%%%%%%%%%%%%%%%%%%%%%%%%%%%%


 \pagenumbering{roman}
 \setcounter{page}{1}
 \bstctlcite{IEEEexample:BSTcontrol}

 \tableofcontents
 \listoffigures
 \listoftables
 \clearpage
 
 \pagenumbering{arabic}
 \setcounter{page}{1}

 %%%%%%%%%%%%%%%%%%%%%%%%%%%%%%%%%%%%%%%%
 % Capítulos                            %
 %%%%%%%%%%%%%%%%%%%%%%%%%%%%%%%%%%%%%%%%

 % SPDX-FileCopyrightText: 2025 Jorge Teixeira Crespo <jorge.teixeira@udc.es>
%
% SPDX-License-Identifier: GPL-2.0-or-later

\chapter{Introduction}
\label{chap:introducion}

\section{Context and Motivation}

GPUL (Grupo de Programadores e Usuarios de Linux) \cite{gpul_web} is a non-profit association founded in 1998 in A Coruña, dedicated to promoting Free and Open Source Software (FOSS) through various community activities, including workshops, hackathons, and technical conferences. Over the years, GPUL has become an essential part of the Galician and broader Spanish FOSS communities, helping introduce students and young professionals to real-world software development and community engagement.

Currently, GPUL's infrastructure is spread across two legacy servers, GPULINO and GPULON. These servers host critical services, including an outdated Nextcloud instance containing vital administrative documents and historical content, and a Mailman 2 instance managing essential mailing lists. Despite their critical role, both servers suffer from severe maintenance issues, instability, and lack of proper backups. These problems are amplified by frequent turnover among the board members every two years, which creates significant knowledge gaps due to insufficient documentation. As the current president, I have directly faced challenges in maintaining this infrastructure, which limits our ability to reliably host essential services and forces us to rely on external providers during significant events, thereby incurring unnecessary expenses.

Practical examples highlight the severity of these issues. Our email setup uses redirects for critical addresses under the gpul.org domain, leading to our emails frequently being marked as spam. This complicates event organization, affecting communication with sponsors and participants. Another instance occurred when our Nextcloud server became unresponsive just before the financial quarter deadline due to a full disk caused by unrotated logs. With no documentation and inadequate tools, it took four days of communication with our hosting provider and former administrators to resolve the issue. This incident brought us dangerously close to missing important administrative deadlines, potentially resulting in fines.

\section{Objectives}

Given these critical challenges, the objectives of this project are to:

\begin{itemize}
    \item Modernize and simplify GPUL's infrastructure to ensure long-term maintainability.
    \item Replace outdated services (Mailman 2 and legacy Nextcloud) with modern, modular open-source solutions.
    \item Implement proper documentation to facilitate knowledge transfer and reduce dependency on individual members.
    \item Enhance the reliability, performance, and security of critical infrastructure services, incorporating robust backup solutions.
    \item Empower new contributors with clear processes and documentation, ensuring continuity and sustainability.
    \item Ensure alignment with GPUL's foundational values of openness and community-driven development.
\end{itemize}

The priority lies in modernization, simplification, and thorough documentation, underpinned by improved security and reliable backup procedures.

\section{Existing Solutions}

In evaluating potential solutions, we considered both fully managed SaaS platforms and integrated suites. While managed services generally conflict with our goal of self-hosting or impose significant licensing costs, we have decided to implement Odoo \cite{odoo_web} specifically for its invoicing capabilities, which are particularly valuable given Spain's new \gls{verifactu} requirements \cite{BOE-A-2024-22138}. While we are currently focusing on the invoicing module, we maintain the flexibility to adopt additional Odoo modules in the future if they prove beneficial. For other services, we continue with our modular approach, upgrading existing open-source solutions such as transitioning from Mailman 2 to Mailman 3 integrated with Fedora's HyperKitty \cite{hyperkitty_web} for better usability, and updating Nextcloud to a recent and stable version. For critical services like email, we are exploring modern self-hosted setups but remain open to outsourcing this component due to its crucial role and complexity.

Thus, the customized renewal of GPUL's infrastructure, grounded in open-source principles and tailored specifically to our community's practical requirements and limitations, emerges as the best solution.

\section{Thesis Structure}

Following this introduction, the thesis will provide a detailed inventory and analysis of GPUL's existing infrastructure, propose a new infrastructure architecture, evaluate and select appropriate tools and hosting solutions, detail the migration strategy and execution, and conclude by providing comprehensive documentation for future maintainers.

 % SPDX-FileCopyrightText: 2025 Laura Milagros Castro Souto <lcastro@udc.es>
%
% SPDX-License-Identifier: GPL-2.0-or-later

\chapter{Contido demostrativo}
\label{chap:demo}

\lettrine{E}{ntre} a introdución e as conclusións, o documento conterá
tantos capítulos como sexa preciso, sempre con coidado de non rebasar
o límite de 80 páxinas fixado polo regulamento de TFGs.

Empregaremos éste de xeito demostrativo, para ilustrar o uso de
elementos habituais que poidan ser de utilidade\footnote{Por exemplo,
  isto é unha nota a pé de páxina.}.

\section{Inclusión de imaxes}

Se precisamos imaxes no noso documento, incluirémolas do xeito que se
indica na figura~\ref{fig:exemplo} (páxina~\pageref{fig:exemplo}). Se
o facemos así, \LaTeX ubicará cada imaxe no mellor lugar posible,
lugar que pode variar a medida que o documento vaia crecendo coa
inclusión de máis texto e outros elementos (máis imaxes, táboas,
etc.).

\begin{figure}[hp!]
  \centering
  \includegraphics[width=0.75\textwidth]{imaxes/udc.png}
  \caption{Pé de imaxe descritivo.}
  \label{fig:exemplo}
\end{figure}

Recoméndase almacenar os ficheiros gráficos no directorio
\texttt{imaxes}.

\subsection{Inclusión de varias sub-imaxes}

Se precisamos inserir imaxes relacionadas, pode ser apropiado
incluílas como sub-figuras, do xeito que se pode apreciar na
figura~\ref{fig:exemplo-subfiguras} (páxina~\pageref{fig:exemplo-subfiguras})
coas imaxes~\ref{fig:subfigura-rotada} e~\ref{fig:subfigura-deformada}.
Como se pode ver nos exemplos desta sección, sempre é recomendable
referirse ás imaxes (ou táboas e outros elementos \emph{flotantes},
que se demostrarán nas seccións seguintes deste capítulo demostrativo)
pola súa referencia, xa que dese xeito non dependemos de onde
queden ubicados os elementos en cuestión.

\begin{figure}[hp!]
  \centering
  \begin{subfigure}[c]{0.3\textwidth}
    \includegraphics[angle=45,width=\textwidth]{imaxes/udc.png}
    \caption{Pé de subimaxe rotada.}
    \label{fig:subfigura-rotada}
  \end{subfigure}
  \hspace{0.1\textwidth}
  \begin{subfigure}[c]{0.3\textwidth}
    \includegraphics[width=\textwidth,height=3cm]{imaxes/udc.png}
    \caption{Pé de subimaxe deformada.}
    \label{fig:subfigura-deformada}
  \end{subfigure}
  \caption{Pé de imaxe xeral.}
  \label{fig:exemplo-subfiguras}
\end{figure}

\section{Inclusión de táboas}

Se precisamos táboas no noso documento, incluirémolas do xeito que se
indica na táboa~\ref{tab:exemplo} (páxina~\pageref{tab:exemplo}). Se
o facemos así, \LaTeX ubicará cada táboa no mellor lugar posible,
lugar que pode variar a medida que o documento vaia crecendo coa
inclusión de máis texto e outros elementos (máis imaxes, táboas,
etc.).

\begin{table}[hp!]
  \centering
  \rowcolors{2}{white}{udcgray!25}
  \begin{tabular}{c|c}
  \rowcolor{udcpink!25}
  \textbf{Título de columna} & \textbf{Outro título de columna} \\\hline
  \textit{Título de fila} & Contido da cela \\
  \textit{Título de fila} & Contido da cela \\
  \textit{Título de fila} & Contido da cela \\
  \textit{Título de fila} & Contido da cela \\
  \textit{Título de fila} & Contido da cela \\
  \textit{Título de fila} & Contido da cela \\
  \end{tabular}
  \caption{Pé de táboa descritivo.}
  \label{tab:exemplo}
\end{table}

\subsection{Inclusión de táboas longas}

Para táboas longas que ocupan varias páxinas, como é o caso da \ref{tab:longa}
(páxina~\pageref{tab:longa}), recoméndase o uso do paquete \texttt{lontable},
incluído xa entre os paquetes recomendados no ficheiro raíz do proxecto
(\verb+memoria_tfg.tex+).

\rowcolors{2}{white}{udcgray!25}
\begin{longtable}{l|r|c}
  \caption{Pé descritivo dunha táboa longa}
  \label{tab:longa} \\

  \rowcolor{udcpink!25}
  \textbf{Primeira columna} & \textbf{Segunda columna} & \textbf{Terceira columna} \\\hline
  \endfirsthead

  \multicolumn{3}{c}{\tablename\ \thetable{} -- {\small \textit{(vén da páxina anterior)}}} \\
  \rowcolor{udcpink!25}
  \textbf{Primeira columna} & \textbf{Segunda columna} & \textbf{Terceira columna} \\\hline
  \endhead

  \multicolumn{3}{c}{\dotfill{\small \textit{(continúa na páxina seguinte)}}\dotfill} \\
  \endfoot

  \endlastfoot

  Texto de exemplo & abcdef ghjijklmn & 123.456778 \\
  Texto de exemplo & abcdef ghjijklmn & 123.456778 \\
  Texto de exemplo & abcdef ghjijklmn & 123.456778 \\
  Texto de exemplo & abcdef ghjijklmn & 123.456778 \\
  Texto de exemplo & abcdef ghjijklmn & 123.456778 \\
  Texto de exemplo & abcdef ghjijklmn & 123.456778 \\
  Texto de exemplo & abcdef ghjijklmn & 123.456778 \\
  Texto de exemplo & abcdef ghjijklmn & 123.456778 \\
  Texto de exemplo & abcdef ghjijklmn & 123.456778 \\
  Texto de exemplo & abcdef ghjijklmn & 123.456778 \\
  Texto de exemplo & abcdef ghjijklmn & 123.456778 \\
  Texto de exemplo & abcdef ghjijklmn & 123.456778 \\
  Texto de exemplo & abcdef ghjijklmn & 123.456778 \\
  Texto de exemplo & abcdef ghjijklmn & 123.456778 \\
  Texto de exemplo & abcdef ghjijklmn & 123.456778 \\
  Texto de exemplo & abcdef ghjijklmn & 123.456778 \\
  Texto de exemplo & abcdef ghjijklmn & 123.456778 \\
  Texto de exemplo & abcdef ghjijklmn & 123.456778 \\
  Texto de exemplo & abcdef ghjijklmn & 123.456778 \\
  Texto de exemplo & abcdef ghjijklmn & 123.456778 \\
  Texto de exemplo & abcdef ghjijklmn & 123.456778 \\
  Texto de exemplo & abcdef ghjijklmn & 123.456778 \\
  Texto de exemplo & abcdef ghjijklmn & 123.456778 \\
  Texto de exemplo & abcdef ghjijklmn & 123.456778 \\
  Texto de exemplo & abcdef ghjijklmn & 123.456778 \\
  Texto de exemplo & abcdef ghjijklmn & 123.456778 \\
  Texto de exemplo & abcdef ghjijklmn & 123.456778 \\
  Texto de exemplo & abcdef ghjijklmn & 123.456778 \\
  Texto de exemplo & abcdef ghjijklmn & 123.456778 \\
  Texto de exemplo & abcdef ghjijklmn & 123.456778 \\
  Texto de exemplo & abcdef ghjijklmn & 123.456778 \\
  Texto de exemplo & abcdef ghjijklmn & 123.456778 \\
  Texto de exemplo & abcdef ghjijklmn & 123.456778 \\
  Texto de exemplo & abcdef ghjijklmn & 123.456778 \\
  Texto de exemplo & abcdef ghjijklmn & 123.456778 \\
  Texto de exemplo & abcdef ghjijklmn & 123.456778 \\
  Texto de exemplo & abcdef ghjijklmn & 123.456778 \\
  Texto de exemplo & abcdef ghjijklmn & 123.456778 \\
  Texto de exemplo & abcdef ghjijklmn & 123.456778 \\

\end{longtable}


\subsection{Inclusión de táboas con celas que ocupan varias columnas ou filas}

En ocasións pode resultar de interese incluír nunha táboa unha cela que se estenda
a través de varias columnas, como ocorre na táboa~\ref{tab:exemplocolumnas}
(páxina~\pageref{tab:exemplocolumnas}).

\begin{table}[hp!]
  \centering
  \rowcolors{2}{white}{udcgray!25}
  \begin{tabular}{c|c|c}
  \rowcolor{udcpink!25}
  \multicolumn{3}{c}{\textbf{Cela en varias columnas}} \\\hline
  \rowcolor{udcpink!25}
  \textbf{Título de columna} & \textbf{Outro título de columna} & \textbf{Outro título máis} \\\hline
  \textit{Título de fila}    & Contido da cela                  & Contido da cela \\
  \textit{Título de fila}    & Contido da cela                  & Contido da cela \\
  \textit{Título de fila}    & \multicolumn{2}{c}{Contido da cela múltiple} \\
  \textit{Título de fila}    & Contido da cela                  & Contido da cela \\
  \end{tabular}
  \caption{Pé de táboa descritivo (táboa con celas que ocupan varias columnas).}
  \label{tab:exemplocolumnas}
\end{table}

Tamén pode resultar necesario facer o propio mais en varias filas da mesma columna,
como ocorre na táboa~\ref{tab:exemplofilas} (páxina~\pageref{tab:exemplofilas}).
Para isto é preciso o paquete \texttt{multirow}, incluído entre os recomendados no
ficheiro raíz do proxecto (\verb+memoria_tfg.tex+).

O uso de celas multifila requerirá do xuste da coloración das filas, a fin de manter
a coherencia entre o contido e o continente. Así, no canto de usar un único comando
\verb+rowcolors+ para indicar a alternancia en toda a táboa, usaremos o comando
\verb+rowcolor+ antes dunha fila que queiramos colorear, e o comando \verb+cellcolor+
dentro dunha cela que queiramos colorear.

\begin{table}[hp!]
  \centering
  \begin{tabular}{c|c}
  \rowcolor{udcpink!25}
  \textbf{Título de columna} & \textbf{Outro título de columna} \\\hline
  \multirow{2}{*}{\textit{Título de fila}} & \cellcolor{udcgray!25} Contido da cela \\
                                           & Contido da cela \\
  \rowcolor{udcgray!25}
  \textit{Título de fila}                  & Contido da cela \\
  \multirow{3}{*}{\textit{Título de fila}} & Contido da cela \\
                                           & \cellcolor{udcgray!25} Contido da cela \\
                                           & Contido da cela \\
  \rowcolor{udcgray!25}
  \textit{Título de fila}                  & Contido da cela \\
  \end{tabular}
  \caption{Pé de táboa descritivo (táboa con celas que ocupan varias filas).}
  \label{tab:exemplofilas}
\end{table}

Por suposto, pódense combinar nunha mesma táboa os dous tipos de celas (as que se 
estenden máis dunha fila e máis dunha columna), como na táboa~\ref{tab:exemplofilasecolumnas}
(páxina~\pageref{tab:exemplofilasecolumnas}).

\begin{table}[hp!]
  \centering
  \begin{tabular}{c|c|c}
  \rowcolor{udcpink!25}
  \multicolumn{3}{c}{\textbf{Cela en varias columnas}} \\\hline
  \rowcolor{udcpink!25}
  \textbf{Título de columna}               & \textbf{Outro título de columna}             & \textbf{Outro título máis} \\\hline
  \multirow{2}{*}{\textit{Título de fila}} & \cellcolor{udcgray!25} Contido da cela       & \cellcolor{udcgray!25} Contido da cela \\
                                           & Contido da cela                              & Contido da cela \\
  \rowcolor{udcgray!25}
  \textit{Título de fila}                  & \multicolumn{2}{c}{Contido da cela múltiple} \\
  \multirow{3}{*}{\textit{Título de fila}} & Contido da cela                              & Contido da cela \\
                                           & \multicolumn{2}{c}{\cellcolor{udcgray!25} Contido da cela múltiple} \\
                                           & \multicolumn{2}{c}{Contido da cela múltiple} \\
  \rowcolor{udcgray!25}
  \textit{Título de fila}                  & Contido da cela                              & Contido da cela \\
  \end{tabular}
  \caption{Pé de táboa descritivo (táboa con celas que ocupan varias columnas).}
  \label{tab:exemplofilasecolumnas}
\end{table}

\section{Inclusión de código fonte}

Se precisamos incluír fragmentos de código fonte, podemos facelo, por exemplo, da
seguinte maneira:

\begin{lstlisting}[language=C]
#include <stdio.h>
#define N 10

int main()
{
  int i;

  // Isto é un comentario
  puts("Ola, mundo!");

  for (i = 0; i < N; i++)
  {
    puts("LaTeX é a ferramenta de edición ideal para profesionais da informática!");
  }

  return 0;
}
\end{lstlisting}

\section{Uso da relación de acrónimos e do glosario}

Os acrónimos edítanse no ficheiro \texttt{bibliografia/acronimos.tex}
e úsanse empregando a orde \texttt{acrlong} para obter o termo
completo (deste xeito: \acrlong{erlang}), a orde \texttt{acrshort}
para obter o acrónimo (deste xeito: \acrshort{erlang}). A primeira vez
que usamos un termo con acrónimo no documento é recomendable usar orde
\texttt{acrfull} (que produce ambas versións á vez:
\acrfull{erlang}). Os acrónimos que non se usan no documento, non
aparecen na relación que se xerar na versión PDF.

Pola súa banda, os termos do glosario edítanse no ficheiro
\texttt{bibliografia/glo\-sa\-rio.tex} e úsanse empregando a orde
\texttt{gls} (deste xeito, \gls{bytecode}) ou \texttt{Gls} (deste
xeito, \Gls{bytecode}). Ao igual que os acrónimos, os termos que non
se usan no documento, non aparecen na relación que se xera na versión
PDF.

%\include{contido/...}
 % SPDX-FileCopyrightText: 2025 Laura Milagros Castro Souto <lcastro@udc.es>
%
% SPDX-License-Identifier: GPL-2.0-or-later

\chapter{Conclusións}
\label{chap:conclusions}

\lettrine{D}{erradeiro} capítulo da memoria, onde se presentará a
situación final do traballo, as leccións aprendidas, a relación coas
competencias da titulación en xeral e a mención en particular,
posibles liñas futuras,\dots

\Blindtext


 %%%%%%%%%%%%%%%%%%%%%%%%%%%%%%%%%%%%%%%%
 % Apéndices, glosarios e bibliografía  %
 %%%%%%%%%%%%%%%%%%%%%%%%%%%%%%%%%%%%%%%%

 \appendix
 \appendixpage
 % SPDX-FileCopyrightText: 2025 Laura Milagros Castro Souto <lcastro@udc.es>
%
% SPDX-License-Identifier: GPL-2.0-or-later

\chapter{Material adicional}
\label{chap:adicional}

\lettrine{E}{xemplo} de capítulo con formato de apéndice, onde se pode
incluír material adicional que non teña cabida no corpo principal do
documento, suxeito á limitación de 80 páxinas establecida no
regulamento de TFGs.

\Blindtext

%\include{anexos/...}

 \printglossary[type=\acronymtype,title=\nomeglosarioacronimos]
 \printglossary[title=\nomeglosariotermos]

 \bibliographystyle{IEEEtranN}
 \bibliography{\bibconfig,bibliografia/bibliografia}
 \clearpage
 
\end{document}

%%%%%%%%%%%%%%%%%%%%%%%%%%%%%%%%%%%%%%%%%%%%%%%%%%%%%%%%%%%%%%%%%%%%%%%%%%%%%%%%
