\chapter{Contido demostrativo}
\label{chap:demo}

\lettrine{E}{ntre} a introdución e as conclusións, o documento conterá
tantos capítulos como sexa preciso, sempre con coidado de non rebasar
o límite de 80 páxinas fixado polo regulamento de TFGs.

Empregaremos éste de xeito demostrativo, para ilustrar o uso de
elementos habituais que poidan ser de utilidade.

\section{Inclusión de imaxes}

Se precisamos imaxes no noso documento, incluirémolas do xeito que se
indica na figura~\ref{fig:exemplo} (páxina~\pageref{fig:exemplo}). Se
o facemos así, \LaTeX ubicará cada imaxe no mellor lugar posible,
lugar que pode variar a medida que o documento vaia crecendo coa
inclusión de máis texto e outros elementos (máis imaxes, táboas,
etc.).

\begin{figure}[hp!]
  \centering
  \includegraphics[width=0.75\textwidth]{imaxes/udc.png}
  \caption{Pé de imaxe descritivo}
  \label{fig:exemplo}
\end{figure}

Recoméndase almacenar os ficheiros gráficos no directorio
\texttt{imaxes}.

