% SPDX-FileCopyrightText: 2025 Jorge Teixeira Crespo <jorge.teixeira@udc.es>
%
% SPDX-License-Identifier: GPL-2.0-or-later

\chapter{Introduction}
\label{chap:introduction}

\lettrine{I}{n} the current digital era, almost every organization relies on a robust technological infrastructure. Day-to-day tasks---from coordination and communication to document management and legal compliance---depend on online or digitally managed tools, making IT infrastructure essential.

For non-profit associations and student communities, the challenge is particularly complex. These entities often lack dedicated IT staff and operate with minimal budgets, yet they are expected to manage technological services. Moreover, many of them, especially those driven by ideological commitments, reject proprietary or \gls{saas} solutions in favor of \gls{foss}. This decision aligns with their values but often increases the burden of system maintenance and security.

A typical association today might need to run one or more websites, handle mailing lists, share files securely, issue invoices under national tax regulations, and coordinate internal communication. These needs must be met through tools that are easy to maintain, well-documented, and adaptable to the association's size and structure. However, in practice, infrastructure is often built incrementally over many years by different people, leading to heterogeneous, outdated, and fragile systems.

This work addresses these challenges by analyzing the typical software needs of non-profit associations and proposing practical guidelines for building and maintaining the required infrastructure. In addition to identifying common services and technical requirements, we also study the case of a long-standing real-world association, which has been operating for more than 25 years. This provides an opportunity to examine the problems that often arise when infrastructure grows organically over time: ad-hoc patches, short-term decisions that eventually become permanent, and the resulting accumulation of technical debt. We discuss the process of transitioning from such a fragmented system to a more rational, maintainable, and future-proof infrastructure.


\section{Context and motivation}

GPUL (Grupo de Programadores e Usuarios de Linux) \cite{gpul-web} provides a concrete example of such an organization, with characteristics that make its case especially notable. Founded in 1998 in A Coruña, GPUL promotes Free Software and technical education through community events, workshops, and conferences. Closely tied to the field of computer engineering and to the Faculty of Computer Science at the Universidade da Coruña, many of its members are current or former students, as well as professionals, with a strong technical background. This makes it easier to adopt and maintain open-source infrastructure, and also turns the infrastructure itself into a valuable learning environment. The association's long-standing commitment to the principles of software freedom drives it to self-host most of its services using libre tools, an approach that aligns with its values but also amplifies the common challenges of volunteer-managed systems.

Today, GPUL's services are distributed across two aging servers, GPULINO and GPULON, which run critical tools like an old Nextcloud instance for internal documentation and a Mailman 2 server that manages our mailing lists. These machines suffer from poor documentation, lack of backups, and limited monitoring, issues that are worsened by the fact that the board of the association changes every two years, often leaving gaps in technical knowledge. As current president, we have had to deal with several crises caused by this fragile setup.

For instance, misconfigured email redirection caused emails to be flagged as spam, and a full disk on Nextcloud nearly made us miss administrative deadlines.

Faced with this situation, we began a complete renewal of GPUL's infrastructure. The goal is to modernize services while keeping them well documented and sustainable. Managed services and commercial tools were mostly ruled out for cost or philosophical reasons. For invoicing we selected FacturaScripts\cite{facturascripts-antifraude}, a Spanish open source solution compatible with the new VeriFactu regulations\cite{boe-a-2024-22138}. The rest of the stack revolves around upgraded \gls{foss} tools: migrating from Mailman 2 to Mailman 3 with HyperKitty\cite{hyperkitty-web}, a modern Nextcloud, and improved backups and monitoring. We continue to evaluate whether email should remain self-hosted or move to a trusted provider. This thesis documents the entire renewal process, aiming to leave a maintainable and resilient infrastructure for future teams.

\section{Objectives}

Given these critical challenges, the objectives of this project are to:

\begin{itemize}
    \item Modernize and simplify GPUL's infrastructure to ensure long-term maintainability.
    \item Replace outdated services (like Mailman 2 and legacy Nextcloud) with modern, modular open-source solutions.
    \item Implement proper documentation to facilitate knowledge transfer and reduce dependency on individual members.
    \item Enhance the reliability, performance, and security of critical infrastructure services, incorporating robust backup solutions.
    \item Empower new contributors with clear processes and documentation, ensuring continuity and sustainability.
    \item Ensure alignment with GPUL's foundational values of openness and community-driven development.
\end{itemize}

The priority lies in modernization, simplification, and thorough documentation, underpinned by improved security and reliable backup procedures.

\section{Thesis Structure}

This thesis is organized as follows:
\begin{itemize}
    \item Chapter~\ref{chap:project-methodology}: project methodology.
    \item Chapter~\ref{chap:current-infrastructure}: analysis of the existing infrastructure.
    \item Chapter~\ref{chap:state-of-the-art}: review of available solutions.
    \item Chapter~\ref{chap:technology-selection}: technological decisions.
    \item Chapter~\ref{chap:hosting-provider}: hosting provider selection.
    \item Chapter~\ref{chap:implementation}: implementation details.
    \item Chapter~\ref{chap:conclusions}: conclusions and future work.
\end{itemize}
