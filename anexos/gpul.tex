% SPDX-FileCopyrightText: 2025 Laura Milagros Castro Souto <lcastro@udc.es>
%
% SPDX-License-Identifier: GPL-2.0-or-later

\chapter{GPUL: More than 25 years freeing minds}
\label{chap:GPUL}

\lettrine{T}{he} Group of Programmers and Users of Linux (GPUL) is a
\gls{glug} in A Coruña (Spain), affiliated with the Faculty of
Computer Science at the University of A Coruña. The group was formally
established in September 1998 as a university association, with bylaws
that defined its main goals as the promotion of Open Source software
concepts and programming based on open standards as a development
philosophy. These initial bylaws also explicitly mentioned programming
for the Linux operating system as one of the association’s objectives.
Over the years, these principles and goals, originally thought with a
more technical perspective in mind, have gradually evolved and
expanded to incorporate an ethical dimension, shifting from the
initial focus on Open Software toward a stronger alignment with the
ideals of Free Software.

From the very beginning, GPUL set four key priorities that would remain
central to its activities as a group committed to its goals. In fact,
the meeting minutes of the first board meeting, held on the same day as its
official founding, state the following agreements:
\begin{itemize}
\item Maintain a server, accessible via the internet, to centralize all
activities and projects of the association.
\item Create and maintain electronic mailing lists for both communication
among members and internal organization.
\item Organize short courses, specialized talks, and trips to major events
where the Linux community gathers.
\item Create and maintain a website containing all GPUL’s content and
activities, serving both as the group’s connection to the broader
Linux community and as an outreach platform.
\end{itemize}

\section{Origins}

The origins of GPUL can be traced back to the \gls{bbs} of the Faculty
of Computer Science at the University of A Coruña, known as
FiCBBS. BBSs were messaging board systems that emerged in the 1980s
and allowed users to exchange messages and files via modem over
regular telephone lines, using a technology called Fido. FidoNet and
the many BBSs that spread worldwide were highly popular at a time when
the internet was not yet widely accessible.

In the early 1990s, the Faculty of Computer Science already had its own
BBS: FiCBBS, which began operating in 1992. The growing popularity of the
GNU/Linux operating system (though the early GPUL members admit they
weren’t fully aware of what GNU, the Free Software Foundation, or the
Free Software movement actually were, and simply referred to it as Linux)
and the arrival of the first distributions that made it somewhat easier
to install on a personal computer had a strong impact on many FiCBBS users.

By 1996, users were actively exchanging messages on the fic.linux area
of FiCBBS, organizing meetups to tinker with the new system, and
placing group orders for CDs containing the most popular distributions
(they were difficult to obtain in that time). This group of enthusiasts
decided to form an association centered around this new operating system
that had captured their interest. And so, in September 1998, GPUL was
founded in office 0.05 of the Faculty of Computer Science at the UDC.

Although GPUL's founders were initially focused on the purely
technical aspects of Linux and its ecosystem, they soon became
increasingly engaged with the Free Software movement, eventually
becoming a key reference within the network of \gls{glug}s in
Spain. Many of them began contributing to major Free and Open Source
projects, such as Debian and the Free Software Foundation, and also
laid the foundations for the Galician Free Software localization
project: Trasno~\cite{Trasno}.

The history of GPUL, and of many of its key members, has been
documented through a series of interviews compiled in the video series
titled ``GPUL, historia de un LUG cualquiera'' (``GPUL, Story of an
Ordinary LUG'' \cite{GPULserie}. These interviews were recorded on the
occasion of the group’s move from its long-standing office to a new
shared association space within the Faculty.


\section{Infrastructure}

From the start, both FiCBBS and GPUL relied on the infrastructure of
the Faculty’s University Extension Committee, particularly its server
known as CEU.  While GPUL lacked its own hardware, it hosted its first
mailing lists there.  Soon, the group acquired its first server, a
typical custom-built PC of the time, which they named Pule.

Initially, GPUL still accessed the internet through CEU, which also
managed email accounts for the board members. Gradually, however, its
services, including the website, which was initially static but soon
evolved into a popular weblog, and the mailing lists, were migrated to
Pule. Not long after, Pule was also assigned its own public IP
address, alongside that of CEU.

When Pule began experiencing hardware issues, it was replaced by
Morpheo, a more professional-grade server. Over time, however, this
was superseded by the group’s first cloud-based server:
GPULINO. Hardware failures and frequent outages in the new shared
association space prompted the migration of GPUL’s services to the
cloud. A little later, a second online server, GPULÓN, was added to
strengthen the infrastructure.

Over time, a wide range of services were deployed on these two cloud
servers, often in response to immediate needs and without much prior
planning. Many of these services were not consistently maintained,
leading to the current situation: an infrastructure that is oversized
in some aspects, obsolete in many others, poorly managed and
documented, and increasingly difficult to maintain.
