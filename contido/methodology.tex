% SPDX-FileCopyrightText: 2025 Jorge Teixeira Crespo <jorge.teixeira@udc.es>
%
% SPDX-License-Identifier: GPL-2.0-or-later

\chapter{Project Methodology}
\label{chap:project-methodology}

\lettrine{T}{ransparency}, reproducibility, and effective collaboration through open tools and platforms form the foundation of this thesis's development methodology. Given the diverse schedules and availability constraints of those involved, asynchronous communication has been explicitly chosen as the primary mode of interaction. This approach facilitates flexibility, allows thorough peer review, and accommodates the varying responsibilities of project participants.

\section{Open Development}

The entire thesis development is conducted openly within a public GitHub repository.\footnote{\url{https://github.com/jorgeteixe/thesis}} All thesis source files, configurations, and relevant documentation are version-controlled, publicly accessible, and managed using a pull request workflow. This enables structured peer review of both technical decisions and thesis content. Continuous integration is automated through GitHub Actions, which generate the thesis PDF document upon every change. This approach ensures immediate feedback, reproducibility, and simplifies collaboration with supervisors.

Throughout the thesis, all configurations, scripts, and tools subject to being open-sourced are either already publicly available or will be made publicly available.\footnote{\url{https://github.com/gpul-org}} Adopting Infrastructure as Code (IaC) principles wherever possible ensures transparency, reproducibility, and ease of maintenance. Additionally, official documentation will be openly hosted on the association's website, providing comprehensive guidance and ensuring traceability and accountability for all project-related decisions. This open and documented approach not only supports transparency but also creates a valuable resource for future students, contributors, and maintainers.

% TODO: actual project methodology: approving in board meeting, analysis of current state, decision making, implementation plan, board approval, actual implementation.
\section{Project Planning}

TBD
