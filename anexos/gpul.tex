% SPDX-FileCopyrightText: 2025 Emilio José Padrón González <emilio.padron@udc.es>
%
% SPDX-License-Identifier: GPL-2.0-or-later

\chapter{GPUL: +25 Years Freeing Minds}

\lettrine{S}{ince} its founding in September~1998, the Group of Programmers
and Users of Linux (GPUL) \cite{gpul-web} has been a driving force for Free Software in A~Coru\~na.
Rooted in the Faculty of Computer Science at the University of
A Coru\~na, this \gls{glug} began as a student initiative to promote open
standards and the GNU/Linux ecosystem. The original bylaws highlighted
the importance of open development and explicitly named Linux as one of
its key areas of activity. Over time those purely technical objectives
expanded to embrace a strong ethical commitment to the ideals of Free
Software.

From the outset, GPUL adopted four guiding principles, documented in the
minutes of its first board meeting:
\begin{itemize}
  \item Maintain a server accessible through the internet to centralize
    all activities and projects.
  \item Create and sustain mailing lists for internal organization and
    member communication.
  \item Organize short courses, technical talks and trips to major
    community events.
  \item Develop and maintain a website that both disseminates the
    association's work and connects it to the wider community.
\end{itemize}

\section{Origins}

GPUL's roots go back to the Faculty's \gls{bbs}, FiCBBS, which began
operating in 1992. Bulletin Board Systems such as this allowed users to
exchange messages and files via modem long before widespread internet
access \cite{Fido93}. By the mid--1990s a small but enthusiastic group on the
\texttt{fic.linux} area of FiCBBS was experimenting with the early
GNU/Linux distributions, coordinating meetups and even bulk ordering CDs
to ease installation. The momentum of that community culminated in the
formal creation of GPUL in office~0.05 of the Faculty.

Although the founding members were initially focused on the technical
side of Linux, they soon engaged with the broader Free Software
movement. Many became contributors to projects such as Debian and the
Free Software Foundation, and they also set the stage for
Trasno~\cite{Trasno}, the Galician localization community. The group's
history, as well as the contributions of many of its members, is
documented in the video series ``GPUL, historia de un LUG
cualquiera''~\cite{GPULserie}, recorded when the association moved from
its long-standing office to a new shared space.

\section{Infrastructure}

Both FiCBBS and the newborn GPUL initially relied on the Faculty's
University Extension Committee server (CEU) for connectivity. While the
group lacked dedicated hardware, it hosted its first mailing lists there.
Soon a home-built PC, affectionately named \emph{Pule}, became GPUL's
first server. Gradually services such as the website---which started as
static pages and grew into a popular weblog---and the mailing lists were
migrated from CEU to Pule, which eventually obtained its own public IP
address.

Pule was eventually replaced by a more robust machine, \emph{Morpheo},
and later by the association's first cloud-based host, \emph{GPULINO}.
Frequent outages in the newly shared association space motivated the
move to the cloud. A second server, \emph{GPUL\'ON}, soon followed to
reinforce the infrastructure. Over the years numerous services were
deployed on these two hosts, often in an ad-hoc manner. Many were left
without proper maintenance or documentation, leading to an oversized and
outdated infrastructure that is increasingly difficult to sustain.
